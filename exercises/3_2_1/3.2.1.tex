\documentclass[12pt]{article}
\usepackage{amsmath}
\begin{document}

Exercise $3.2.1$\\

(a) Show that if $Q$ is orthogonal,then $Q^{-1}$ is orthogonal. (b) Show that if $Q_1$ and $Q_2$ are othogonal, then $Q_1Q_2$ is orthogonal.

\noindent\rule{\textwidth}{1pt}

Answer\\

a) By definition: 
$$Q^tQ = I$$
is orthogonal.\\
Inverting both sides: 
$$(Q^t)^{-1}Q^{-1} = I^{-1}$$
Using the orthogonal matrices property $Q^t = Q^{-1}$:
$$(Q^{-1})^tQ^{-1} = I^{-1}$$
$$(Q^{-1})^tQ^{-1} = I$$
Therefore, $Q^{-1}$ is orthogonal, from the same definition.\\

b)Show that $(Q_1Q_2)^t(Q_1Q_2) = I$
\begin{equation*}
\begin{gathered}
Q_1^tQ_1 = I\\
Q_1^tQ_1Q_2 = Q_2\\
Q_2^tQ_1^tQ_1Q_2 = Q_2^tQ_2\\
(Q_2^tQ_1^t)Q_1Q_2 = (Q_2^tQ_2)\\
(Q_1Q_2)^tQ_1Q_2 = I\\
\end{gathered}
\end{equation*}
\end{document}
