\documentclass[11pt]{article}
\usepackage{amsmath}
\begin{document}

Exercise $1.8.1$\\


After $k - 1$ steps of the Gaussian elimination process the coefficient matrix has been transformed to the form
$$
B = 
\begin{bmatrix}
	B_{11}	& B_{12}\\
	0		& B_{22}
\end{bmatrix}
$$
where $B_{11}$ is $(k - 1)$ by $(k-1)$ and upper triangular.\\

Prove that $B$ is singular if the first column of $B_{22}$ is zero. (\textit{Remark:} The fact that $B_{11}$ is upper triangular is of no consequence). \\

\noindent\rule{\textwidth}{1pt}

Answer\\

From definition: If $B$ non-singular, $det B \neq 0$\\

Calculate determinant of $B$ using the coefficient matrix form and applying Laplace along the first column always until reaching line column $k$. Note that doing the same on column $k$ gives \textit{0} times the rest of determinant.
$$
det B = \prod_i^{k-1}a_{ii}\times0
$$
$$
det B = 0
$$ 

Remember that since $B$ is not non-singular, $B$ is singular.
\end{document}
